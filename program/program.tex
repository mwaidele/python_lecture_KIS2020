\documentclass[10pt,a4paper]{article}
\usepackage[latin1]{inputenc}
\usepackage{amsmath}
\usepackage{amsfonts}
\usepackage{amssymb}
\usepackage{graphicx}
\usepackage{geometry}
\usepackage{hyperref}

\geometry{
	inner=27mm,
	outer=24.5mm,
	top=31mm,
	bottom=27mm,
	heightrounded,
	marginparwidth=51pt,
	marginparsep=17pt,
	headsep=24pt
}

\begin{document}
	\title{Python Lecture at KIS 2020}
	\author{Matthias Waidele}
	\date{}
	
	\maketitle
	Please note: This program may be understood as a broad guideline and rough timetable on topics. As such it may be subject to changes before and during the lecture.
	
	\section{Basics}
	20.04. - 24.04. (intensive course, one lecture per day and per topic)
	
		\subsection{Installation, GitLab introduction, \href{https://learn.datacamp.com/}{DataCamp} and First Steps}
		\subsection{Basic Operations, Types and Loops}
		\subsection{Functions, Methods and Scope}
		\subsection{Scripts, Modules, Packages and Good Practices}
		\subsection{Input and Output, Object-oriented Programming, Why Python?}
	
	\vspace{1cm}
	\section{Advanced Topics: Data Structures, Data Handling and Visualization}
	Following weeks (one lecture per week, one to two lectures per topic)
		
		\subsection{Basic Data Structures and Numerics with \texttt{numpy} and \texttt{scipy}}
		\subsection{Visualization with \texttt{matplotlib}}
		\subsection{Advanced Data Structures and Statistics with \texttt{pandas}}
	
	\vspace{1cm}
	\section{Specific Topics}
	Following weeks (one lecture per week, one to two lectures per topic)
		
		\subsection{Advanced Object-oriented Programming, Symbolic calculations with \texttt{sympy}}
		\subsection{Efficient Debugging, Optimization, Parallel Programming (and Cython)}
		\subsection{Utilities for solar physicists with \texttt{fits}, \texttt{h5py}, \texttt{sunpy} and \texttt{astropy}}
%		\subsection{More specific topics ...}
		
	\thispagestyle{empty}
\end{document}